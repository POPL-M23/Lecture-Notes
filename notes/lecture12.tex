\chapter{Lecture 12}
\chapterauthor{Amitabh Paliwal}

\section{Self Reference}

The function refers to itself in its body.

Example:
\begin{lstlisting}
    (define fac 
        (lambda (n)
            (if (= n 0) 
            1
            (* n (fac (sub1 n))))))
\end{lstlisting}
Here fac refers to fac. Also the function in the body is free.

\section{Self Application}
A function is applied to itself.

Eg. Function
\begin{lstlisting}
    (define id
        (lambda (x) x))
\end{lstlisting}

Its self-application
\begin{lstlisting}
    (id id)
\end{lstlisting}

\section{Curried Functions}
Suppose we need to represent a function of function then we write

\begin{lstlisting}
    ((f y) x)
\end{lstlisting}

The same can be represented by

\begin{lstlisting}
    (:f y x)
\end{lstlisting}

where : denotes application.

A curried function will evaluate a sequence of functions each with a single argument.

Eg.

\begin{lstlisting}
    (: f x) = (f x)
    
    (: f x y) = ((f x) y)

    (: f x1 x2 x3) = (((f x1) x2) x3)

    (lambda (x)             =    (lambda: (x y) e)
        (lambda (y) e))
\end{lstlisting}

A curried version of add

\begin{lstlisting}
    (define +:
        (lambda (x)
            lambda (y)
                (+ x y)))
\end{lstlisting}

\section{Redefining factorial}

\begin{lstlisting}
    (define: (G f n)
        (if (= n 0) 1)
            (* n (f (sub1 n))))
\end{lstlisting}

Here the 'f' is bound but in the earlier version of factorial it was free.
It has no recursion.

G takes a function as input. Any function can be passed.

Eg.

\begin{lstlisting}
    (: G add1 3)
    = (* 3 (add1 (sub1 3)))
    = 9

    (: G (lambda(x) (* x x)) 2)
    = (* 2 ((lambda(x) (* x x)) 1))
    = 2

    (: G fac 3)
    = (* 3 (fac (sub1 3)))
    = (* 3 (fac 2))
    = 6
\end{lstlisting}

\section{Self-application to define factorial}

\begin{lstlisting}
    (define: (H f n)
        (if (= n 0)
            1
            (* n ((f f) (sub1 n)))))
\end{lstlisting}

We can say

\begin{lstlisting}
    (: H f n) = (: G (f f) n)
\end{lstlisting}

Getting a feel of H

\begin{lstlisting}
    (: H I 3)
    = (* 3 ((I I) 2))
    = (* 3 (I 2))
    = (* 3 2)
    = 6

    (: H add1 3)
    = (* 3 ((add1 add1) 2))
\end{lstlisting}

But add1 cannot accept a procedure as argument and it throws an error.

H is specific about what can be passed to it. Function that takes a function as input and returns a number. (H H) returns a function on numbers.
We can define this function as a function 'p'.

\begin{lstlisting}
    (define p (H H))

    (p 0)
    = (: H H 0)
    = (if (= 0 0) 1 ...)
    = 1

    (p 1)
    = (: H H 1)
    = (if (= 0 1) 1
            (* 1((H H) 0)))
    = (* 1 1)
    = 1

    (p 2)
    = (* 2 ((H H) 1))
    = (* 2 (p 1))
    = (* 2 1)
    = 2
\end{lstlisting}

p behaves like a factorial function.

Now we can say

\begin{lstlisting}
    (H/G f n) = (: G (f f) n)

Also
    (: H/G H/G n) = (: G (H/G H/G) n)
then
    (H/G H/G) = (G (H/G H/G))
\end{lstlisting}

this is a fixed-point of G. The fixed-point gives us recursion. So we need functions that get us the fixed-point.

\section{Y-combinator}

\begin{lstlisting}
    (Y g) = (let ([s(lambda (x)
                    (g (x x)))]) (s s))

    We can also say,
        (Y g) = (s s) where (s x) = (g (x x))
\end{lstlisting}

Now,

\begin{lstlisting}
    (Y g) = (s s)
          = ((lambda(x)(g (x x))) (lambda(x)(g (x x))))
          = (g (s s))
          = (g (Y g))
\end{lstlisting}

This is a fixed-point which allows us to get recursion.

So, factorial = (Y G).
\\ G is written to extract the factorial recursion using Y.

Example:

\begin{lstlisting}
    (: Y G 2)
    = (: s s 2)  where s = (lambda (x) (G (x x)))
    = ((s s) 2)
    = ((G (s s)) 2)
    = (: G (s s) 2)
    = (if (= 0 2) 1
        (* 2 (: s s 1)))
    = (* 2 (: G (s s) 1))
    = (* 2 (if (= 0 1) 1
            (* 1 (: s s 0))))
    = (* 2 (* 1 (if (= 0 0) 1
                    (* 0 (: G (s s) (sub1 0))))))
    = (* 2  1 1)
    = 2
\end{lstlisting}

This allows us to shrink our language as recursion is handled by
the function application and abstraction.

\subsection{Concern}

Non-termination is a concern in the Y-combinator approach.
In case of lazy-evaluation there is no problem, but an evaluation
strategy like racket's fails. Applicative order strategies lead to
non-terminating sequences.

Example of non-termination

\begin{lstlisting}
    (: Y G 2)
    = (: G (s s) 2))
    = (: G (: G (s s)) 2)
    = (: G (: G (: G (s s) 2)))
\end{lstlisting}
\section{Z-combinator}

This works in applicative order evaluation. Just wrap things in a lambda.

\begin{lstlisting}
    (define (Z g)
        (let ([s (lambda (x)
                (g (lambda (n) (: x x n))))])
            (s s)))

    We can also say:
        (Z g) = (g (lambda (n) (: s s n)))
\end{lstlisting}

Example:

\begin{lstlisting}
    (: Z G 2)
    = (: G (s s) 2)
    = (: G (lambda (n) (: s s n)) 2)
    = (if (= 0 2) 1
        (* 2 (: s s 1)))
    = (* 2 (: s s 1))
    = (* 2 (: G (s s) 1))
    = (* 2 (if (= 0 1) 1 (* 1 (: s s 0))))
    = (* 2 (* 1 (: s s 0)))
    = (* 2 (* 1 (: G s s 0)))
    = (* 2 (* 1 (if (= 0 0) 1 ...)))
    = (* 2 (* 1 1))
    = 2
\end{lstlisting}
