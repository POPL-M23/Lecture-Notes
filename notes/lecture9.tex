\chapter{Lecture 9}
\chapterauthor{Pranav Subramaniam}

% for antecedent-consequent expressions
\newcommand\bigfrac[2]{%
  \begin{array}{c}
    #1 \\
    \hline
    #2
  \end{array}}

\section{Overview of Programming Languages}

\subsection{Mathematical Foundation}
\begin{enumerate}
	\item Before lambda-calculus ("Pre-history" of computing)
	
	\begin{enumerate}
		\item Set Theory
	\end{enumerate}
	\begin{enumerate}
		\item First-Order Logic
	\end{enumerate}
	\begin{enumerate}
		\item Combinatory Logic
	\end{enumerate}
	
	\item Predecessor to Programming Languages \newline
	Provided mathematical framework for programming languages and computing
	
	\begin{enumerate}
		\item Category Theory
		\item $\lambda$ Calculus $\equiv$ Turing Machines (1930s)
		\begin{enumerate}
			\item $\lambda$ Calculus
			\begin{enumerate}
				\item Framework for programming language design
				\item Basis for functional programming
			\end{enumerate}
			\item Turing Machines
			\begin{enumerate}
				\item Computability, complexity, determinism
				\item Related work- Von Neumann architecture, Kurt Gödel's incompleteness theorem
			\end{enumerate}
		\end{enumerate}
	\end{enumerate}
	
\end{enumerate}

\subsection{History}
\begin{table}[H]
  \begin{tabular}{|c|p{2.5cm}|p{2.5cm}|p{2.5cm}|p{2.5cm}|p{2.5cm}|}
    \hline
    \textbf{S No.} & \textbf{Functional Languages} & \textbf{Imperative Languages} & \textbf{OOP} & \textbf{PROLOG (Logic Programming)} & \textbf{Concurrent Distributed Programing Languages} \\
    \hline
    1 & Lisp (1959) & FORTRAN (1957) & Simula- Simulation-oriented, (Dahl and Nygaard - 1996) & Prolog (Colmerauer, 1971) & CSP (Communicating Sequential Processes), (Hoare, 1978) \\
    2 & Landin- SECD Machine (1964) & ALGOL (1968) & CLOS (1988) & & $\Pi$ Calculus, (Milner, 1990s) \\
    3 & Scheme (1977) & Pascal (1970) & C++ (1985) & & \\
    4 & ML (1973) (feature- types) & C (1972) & Java (1985) & & \\
    5 & Haskell (1987) (feature- types) Purely function language & & C\# (2000) & & \\
    6 & Racket & & & \\
    \hline
  \end{tabular}
\end{table}

\section{Closures: Functions as values}
\underline{Higher-order functions} can take functions as arguments and can return functions.

\begin{verbatim}
> (define add
      (lambda (x y)
          (+ x y)))
\end{verbatim}
Add function can be rewritten as-
\begin{verbatim}
> (define add
      (lambda (x) ;;  lambda (x) takes x, returns a function lambda (y)
          (lambda (y)
              (+ x y)))
> (add 3)          
#(procedure>
> ((add 3) 4)
7
\end{verbatim}

Examples of higher-order functions include computing derivatives- ($\frac{d}{dx}$ takes $f$, returns $f'$) and function composition ($h$ = $f$ $\circ$ $g$)\newline\newline
Higher-order functions require the domain for expressible values to contain functions-\newline
\[EXPVAL = NUM \quad | \quad BOOL \quad | \quad FUNCTION\]


\subsection{Functional Language}
Recall the ojective to build a program to compute expressions of numbers. We add the Functional Language to the langauge previously consisting of Arithmetic, IF+DIV, Global and Lexical Scope components.\\
\subparagraph{Abstract Syntax}
\[
e ::=
\begin{array}{ll}
    \overline{n} \qquad \text{ NUM}\\
    \overline{b} \qquad\text{ BOOL}\\
    ifte \qquad e \text{ } e \text{ } e\\
    \lambda \qquad \overline{x} \text{ ... }e\\    
    @ \qquad e \text{ } e \text{ ...}
\end{array}
\]
\newline
Abstract syntax form- (number, boolean, if-then-else, procedure, application)\newline
Notice Arithmetic operators and LET are replaced by $\lambda$ and @.

\subparagraph{Concrete Syntax}
\begin{align}
(let\text{ }([x\text{ }e]\text{ ...}) \textit{ body}) = ((& \lambda\text{ }(x\text{ ...})\nonumber \\&\qquad \textit{body})\nonumber \\&\quad e\text{ ...}) \nonumber
\end{align}

\subsection{Evaluation Semantics}
Expressible values -\newline
\begin{align}
EXPVAL &= NUM \oplus BOOL \oplus FUNCTION \nonumber \\
DENVAL &= EXPVAL \nonumber
\end{align}
\\\underline{NOTE}: Evaluation semantics of a good language design requires precision to ensure bug-free logic and language. Ideally, EXPVAL should be partioned completely and thus should be a disjoint union.\newline
Example: integer datatypes in C allows pointer values. Better design-
\[EXPVAL = NUM \oplus BOOL \oplus POINTER \]

\subparagraph{Rules:}


For grammar $\Gamma$, expression $e$, EXPVAL $v$-
\[\Gamma \vdash e \Rightarrow v \]
($\vdash$: Turnstile operator, expression reads "Under $\Gamma$, e evaluates to v) \newline

\begin{enumerate}
\item $\bigfrac{}{\Gamma \vdash \overline{n} \Rightarrow n}$ NUM
\item $\bigfrac{}{\Gamma \vdash \overline{b} \Rightarrow b}$ BOOL
\item $\bigfrac{\Gamma(x) = v}{\Gamma \vdash x \Rightarrow v}$ ID
\item $\bigfrac{\Gamma \vdash e_{1} \Rightarrow \#t \quad \Gamma \vdash e_{2} \Rightarrow v_{2}}{\Gamma \vdash \textit{ifte } e_{1} \text{ } e_{2} \text{ } e_{3} \Rightarrow v_{2}}$ IFTE-TRUE
\item $\bigfrac{\Gamma \vdash e_{1} \Rightarrow \#f \quad \Gamma \vdash e_{3} \Rightarrow v_{3}}{\Gamma \vdash \textit{ifte } e_{1} \text{ } e_{2} \text{ } e_{3} \Rightarrow v_{3}}$ IFTE-FALSE
\end{enumerate}

Building an application:
$\bigfrac{}{\Gamma \vdash (\lambda \text{ } (x \text{ ... }) \text{ } e) \Rightarrow v}$


\subsection{Lexical Scope vs Dynamic Scope}
Racket code demonstrating lexical scope\newline
\begin{verbatim}
> (let ([x 5])
    (let ([f (lambda (y)   ;; lexical scope
               (+ x y))])
      (let ([x 0])         ;; dynamic scope
        (f 3))))
8
\end{verbatim}

Lexical scope gives returns 8, dynamic scope returns 3.
