
\section{Lecture 3}

\subsection{Properties of abstract reduction systems}

In this lecture, we explore some properties of abstract reduction systems and look at the relationships between them.

\begin{definition}
    An abstract reduction system $( A, \xrightarrow[A]{} )$ is said to be \textbf{Church-Rosser} if $$\forall x, y \in A, \quad x \xleftrightarrow[A]{*} y \implies x \downarrow_A y$$
\end{definition}

Consider abstract reduction systems $A$ and $B$ as shown in the figure below. System $A$ is not Church-Rosser, since $a \xleftrightarrow[A]{*} b$ but $a$ and $b$ are not joinable. System $B$ is a simple example of a Church-Rosser system.

\begin{figure}[htbp]
    \center
    \includegraphics[scale=0.8]{images/lecture1/CR.png}
    \caption{Examples of abstract reduction systems}
\end{figure}


\begin{definition}
    An abstract reduction system $( A, \xrightarrow[A]{} )$ is said to be \textbf{Confluent} if $$\forall a, b, c \in A, \quad a \xrightarrow[A]{*} b \text{ and }  a \xrightarrow[A]{*} c \implies b \downarrow_A c$$
\end{definition}

\begin{definition}
    An abstract reduction system $( A, \xrightarrow[A]{} )$ is said to be \textbf{Semi-confluent} if $$\forall a, b, c \in A, \quad a \xrightarrow[A]{} b \text{ and }  a \xrightarrow[A]{*} c \implies b \downarrow_A c$$
\end{definition}

\begin{theorem}
    For an abstract reduction system, \textbf{Church-Rosser, Confluence and Semi-confluence are equivalent}.
\end{theorem}

\textbf{Proof:} We will prove this theorem in the following three stages. Clearly, the three of them combined result in the theorem stated above.
\begin{enumerate}
    \item $\text{Church-Rosser} \implies \text{Confluence}$
    \item $\text{Confluence} \implies \text{Semi-confluence}$
    \item $\text{Semi-confluence} \implies \text{Church-Rosser}$
\end{enumerate}

First, we will prove that \textbf{if a system is Church-Rosser, it is confluent}. Let $(A, \xrightarrow[A]{})$ be an abstract reduction system that is Church-Rosser. \\
Let $a, b, c \in A : a \xrightarrow[A]{*} b \text{ and } a \xrightarrow[A]{*} c$ \\
By definition, $b \xleftrightarrow[A]{*} c$ \\
Since A is Church-Rosser, $b \xleftrightarrow[A]{*} c \implies b \downarrow_A c$ \\
i.e. $ a \xrightarrow[A]{*} b \text{ and } a \xrightarrow[A]{*} c \implies b \downarrow_A c$, proving that A is confluent


Next, we will prove that \textbf{if a system is Confluent, it is also semi-confluent}. Let $(A, \xrightarrow[A]{})$ be a confluent abstract reduction system. \\
Let $a, b, c \in A : a \xrightarrow[A]{} b \text{ and } a \xrightarrow[A]{*} c$ \\
Since $A \subseteq A^*$, $a \xrightarrow[A]{} b \implies a \xrightarrow[A]{*} b$ \\
Since A is confluent, $a \xrightarrow[A]{*} b \text{ and } a \xrightarrow[A]{*} c \implies b \downarrow_A c$ \\
i.e. $ a \xrightarrow[A]{} b \text{ and } a \xrightarrow[A]{*} c \implies b \downarrow_A c$, proving that A is semi-confluent

Finally, we will prove that \textbf{if a system is semi-confluent, it is also Church-Rosser}. Let $(A, \xrightarrow[A]{})$ be a semi-confluent abstract reduction system. \\
Let $a, b \in A : a \xleftrightarrow[A]{*} b$ \\
Let $p$ be the shortest path connecting $a$ and $b$ in $A^{\leftrightarrow ^*}$. We will use induction on $|p|$ to prove that $a$ and $b$ are joinable. \\
\textbf{Base case:} For $p=0$, we have $a=b$ which makes them trivially joinable. \\
\textbf{Induction step:} Let it be true that if the shortest path connecting $a$ and $b$ in $A^{\leftrightarrow ^*}$ is $|p|$, then $a$ and $b$ are joinable in $A$. We will prove that this is also true for $|p|+1$.

Let $a, b' \in A$ such that the shortest path connecting them in $A^{\leftrightarrow ^*}$ is of length $|p|+1$. Then, $\exists b \in A :$ the shortest path connecting $a$ and $b$ in $A^{\leftrightarrow ^*}$ is $|p|$ and $b \xleftrightarrow[A]{} b'$. Since our induction hypothesis holds true for $|p|$, $a$ and $b$ are joinable (they both reduce to some $c \in A$).


\begin{figure}[htbp]
    \center
    \includegraphics[scale=0.8]{images/lecture1/induction1.png}
    \caption{Abstract reduction system representing the induction step}
\end{figure}
We now have 2 cases. 

\textbf{Case 1:} $b \xleftarrow[A]{} b'$ 

\begin{figure}[htbp]
    \center
    \includegraphics[scale=0.8]{images/lecture1/induction2.png}
    \caption{Abstract reduction system in case 1}
\end{figure}
$b' \xrightarrow[A]{} b \xrightarrow[A]{*} c$. Therefore, $a \downarrow _A b'$. 

\textbf{Case 2:} $b \xrightarrow[A]{} b'$

\begin{figure}[htbp]
    \center
    \includegraphics[scale=0.8]{images/lecture1/induction3.png}
    \caption{Abstract reduction system in case 2}
\end{figure}
Here we have $b \xrightarrow[A]{} b'$ and $b \xleftrightarrow[A]{*} c$. Since $A$ is semi-conflient, $b'$ and $c$ must be joinable. That is, $\exists c' \in A : b' \xrightarrow[A]{*} c' \text{ and } c \xrightarrow[A]{*} c'$.  Since $a \xrightarrow[A]{*} c \xrightarrow[A]{*} c'$, we have $a \xrightarrow[A]{*} c'$. Hence, $a \downarrow _A b'$


In either case, we have shown that $a \downarrow _A b'$, which completes our induction. We have proven that $a \xleftrightarrow[A]{*} b \implies a \downarrow_A b$, which means A is also Church-Rosser.

This completes our proof that Church-Rosser, Confluence and Semi-confluence are equivalent properties of an abstract reduction system. While it may seem redundant to have multiple terms to refer to the same thing, they each give us a different perspective of looking at the same property which can prove to be helpful.


\subsection{Address space}
Data structures provide a way to organize and address data. For a data structure, a valid set of addresses form its address space. This is \textbf{not} a formal definition of address spaces and is only meant to give a broad idea. We will look at an example below to illustrate one way of addressing a binary tree. Let us consider the following addressing of a binary tree: each edge is labelled $1$ or $2$ depending on whether it leads to the left or right descendant of a node; the address of each node is obrained by appending the label of the edge leading into it to the address of its parent, with the root being $\epsilon$. Look at the figure below to bettter understand this.
\begin{figure}[htbp]
    \center
    \includegraphics[scale=0.4]{images/lecture1/address.png}
    \caption{Addressing a binary tree}
\end{figure}

The address space for this binary tree would be the set $S = \{\epsilon, 0, 00, 01, 1, 10, 11 \}$. The set $S_1 = \{ \epsilon, 0, 00, 01, 1 \}$ would also be a valid address space for some binary tree but the set $S_2 = \{ 0, 00, 1 \}$ would not, since it does not contain $\epsilon$, the address of the root.
